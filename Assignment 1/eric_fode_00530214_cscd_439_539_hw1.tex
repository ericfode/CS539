\documentclass[11pt]{article}
\usepackage{fullpage}
\usepackage{graphics,epsfig,color}
\usepackage{wrapfig}
\usepackage{times}
\usepackage{setspace}
\usepackage{amsmath,amsthm,amssymb}
\usepackage{url}
\usepackage{fancyhdr}
\usepackage{enumitem}
\pagestyle{fancy}


\newtheorem{theorem}{Theorem}[section]
\newtheorem{corollary}{Corollary}[section]
\newtheorem{lemma}{Lemma}[section]
\newtheorem{problem}{Problem}
\newtheorem{definition}{Definition}[section]
\newtheorem{observation}{Observation}[section]
\newtheorem{example}{Example}[section]
\newtheorem{openproblem}{Open Problem}[section]
\newtheorem{fact}{Fact}[section]

\newcommand{\qedsymb}{\hfill{\rule{2mm}{2mm}}}
\newenvironment{proofsketch}
{
	\begin{trivlist}
	\item[\hspace{\labelsep}{\noindent Proof Sketch: }]
}{\qedsymb\end{trivlist}}



%the following few lines until usepackage{algorithm2e} is to avoid the
%conflicts of algorithm2e with other packages.
\makeatletter
\newif\if@restonecol
\makeatother
\let\algorithm\relax
\let\endalgorithm\relax
\usepackage[ruled,vlined,linesnumbered]{algorithm2e}

\newcommand{\remove}[1]{}



%--------------------------------


\begin{document}

	\renewcommand{\headrulewidth}{0.4pt}
	\setlength{\headheight}{38.0pt}
	\fancyhead[L]{\bf CSCD359 Homework1, Winter 2012, 
	Eastern Washington University. Cheney, Washington. \\
	\bigskip Name: Eric Fode\hspace{40mm}EWU ID:00530214}


	\noindent{\bf Solution for Problem 1}
	
	let $x = 1$ if the coin is two headed
	$$
	P(x	| h) = \frac{P(h|x) P(x)}{P(h|x) P(x) + P(h|\bar{x}) P(\bar{x})}=
	\frac{2}{3}
	$$
	 Is the probability that the coin is the two headed coin.

	\noindent{\bf Solution for Problem 2}
	Using bayes law the probability can be calculated as such
	$$
	P(sick|positive) = \frac{P(positive|sick) * P(sick)}{(P(positive|sick) * P(sick)		(P(positive|well) * P(well)}
	$$
	which figures out to
	$$
	\frac{.99*.02}{.99*.02+.005*.98} = .8016
	$$
	So the probability that you are sick if the test returns true is about $.80$
	\bigskip
	
	\noindent{\bf Solution for Problem  3}
	\begin{proof} If we craft a graph that all the possible min-cuts require removing two or less
		edges. Then it is simple to see that the maximum number of distinct min-cuts is given by this
		combination
		$$
		n\choose{2} = \frac{n}{(n-2)!2!}
		$$
		if we craft a graph that require removing k edges to create a min-cut we get the combination
		$$
		n\choose{k} = \frac{n}{(n-k)!k!}
		$$
		from this it is simple to see that where k > 2
		$$
		\frac{n}{(n-k)!k!} \leq \frac{n}{(n-2)!2!}
		$$
		and it follow that therefore the graph that you only have to remove two edges has the most min-cuts
		So therefore there exists no graph that has more then 
		$$
		\frac{n(n-1)}{2}
		$$
		min-cut sets
	\end{proof}
	
	\bigskip
	\noindent{\bf Solution for Problem 4}
	$$
	p("proof") = (\frac{1}{25})^5
	p(x) = \frac{1000000-4}{11881376}
	$$
 	$.084$ times.
	\bigskip
	\newpage
	\noindent{\bf Solution for Problem 6}\\
	The expected number of cards that need to be taken from the deck is about is given by
	\begin{align*}	
		X &= \sum\limits_{i=1}^n x_i\\
		E[X] &= E[\sum\limits_{i=1}^n x_i]\\
		E[X] &= \sum\limits_{i=1}^n E[x_i]\\
	\end{align*}
	where $c_i$ is the number of cards that need to be drawn to get the ith distinct card

	$$
	c_i = \left\{
			\begin{array}{ll}
				 1 & \mbox{if the card is new}\\
				 0 & \mbox{if the card has been seen} 
			\end{array}
		\right.
		\\
	$$

	\begin{align*}
		P_i &= Pr(c_i=1)\\
		Pr(c_i=1) &= 1 - \frac{i-1}{n}\\
		E[x_i] &= \frac{1}{p_i}=\frac{n}{n-i+1}\\
	\end{align*}
	Plugging this into the summation we get
	\begin{align*}
		E[X] &= \sum\limits_{i=1}^n \frac{n}{n-i+1}\\
			 &= n\sum\limits_{i=1}^n \frac{1}{i}\\
			 &= n \ln n + n
	\end{align*}
	
	
	The number of unseen cards after $2n$ draws will be
	$$
	n(\frac{n-1}{n})^{2n})
	$$

	\newpage
	\bigskip
	\noindent{\bf Solution for Problem 7}\\
	let $x_i$ be the number of flips to the next head
	\begin{align*}
		E[X] &= E[\sum\limits_{i=1}^k x_i]\\
			 &= \sum\limits_{i=1}^k E[x_i]\\
			 &= \sum\limits_{i=1}^k \frac{1}{p}\\
			 &= \frac{k}{p}
	\end{align*}
	\bigskip
	\noindent{\bf Solution for Problem 8}\\
	\begin{align*}
		E[X] &= E[\sum\limits_{i=1}^{n}|a_i-i|]\\
			 &= \sum\limits_{i=1}^{n}E[|a_i-i|]\\
		Pr(a_i=j)&=\frac{1}{n}\\
		E[|a_i -i|] &= \sum\limits_{j=1}^n \frac{1}{n}j\\
		&= \frac{n+1}{2} - \frac{i}{n}\\
		E[\sum\limits_{i=1}^{n}|a_i-i|]&=\sum\limits_{i=1}^{n}\frac{n+1}{2} - \frac{i}{n}\\
		&=\frac{n(n+1)}{2} - \frac{1}{n}\frac{n(n+1)}{2}\\
		&=\frac{n^2-1}{2}		
	\end{align*}
\end{document}




