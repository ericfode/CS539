\documentclass[11pt]{article}
\usepackage{fullpage}
\usepackage{graphics,epsfig,color}
\usepackage{wrapfig}
\usepackage{times}
\usepackage{setspace}
\usepackage{amsmath,amsthm,amssymb}
\usepackage{url}
\usepackage{fancyhdr}
\usepackage{enumitem}
\pagestyle{fancy}


\newtheorem{theorem}{Theorem}[section]
\newtheorem{corollary}{Corollary}[section]
\newtheorem{lemma}{Lemma}[section]
\newtheorem{problem}{Problem}
\newtheorem{definition}{Definition}[section]
\newtheorem{observation}{Observation}[section]
\newtheorem{example}{Example}[section]
\newtheorem{openproblem}{Open Problem}[section]
\newtheorem{fact}{Fact}[section]

\newcommand{\qedsymb}{\hfill{\rule{2mm}{2mm}}}
\newenvironment{proofsketch}
{
	\begin{trivlist}
	\item[\hspace{\labelsep}{\noindent Proof Sketch: }]
}{\qedsymb\end{trivlist}}



%the following few lines until usepackage{algorithm2e} is to avoid the
%conflicts of algorithm2e with other packages.
\makeatletter
\newif\if@restonecol
\makeatother
\let\algorithm\relax
\let\endalgorithm\relax
\usepackage[ruled,vlined,linesnumbered]{algorithm2e}

\newcommand{\remove}[1]{}



%--------------------------------


\begin{document}

	\renewcommand{\headrulewidth}{0.4pt}
	\setlength{\headheight}{38.0pt}
	\fancyhead[L]{\bf CSCD359 Homework1, Winter 2012, 
	Eastern Washington University. Cheney, Washington. \\
	\bigskip Name: Eric Fode\hspace{40mm}EWU ID:00530214}


	\noindent{\bf Solution for Problem 1}
	$$frac{2}{3}$$ Is the probability that the coin is the two headed coin.
	\newpage
	\noindent{\bf Solution for Problem 2}
	Using bayes law the proabibilty can be calculated as such
	$$
	P(sick|positive) = frac{P(positive|sick) * P(sick)}{(P(positive|sick) * P(sick)(P(positive|well) * P(well)}
	$$
	which figures out to
	$$
	frac{.99*.02}{.99*.02+.005*.98} = .8016
	$$
	So the probibilty that you are sick if the test returns true is about \[ 80% \]
	\bigskip
	
	\noindent{\bf Solution for Problem  3}
	\begin{proof} If we craft a graph that all the possible min-cuts require removing two or less
		edges. Then it is simple to see that the maximum number of distinct min-cuts is givin by this
		combination
		$$
		n\choose{2} = \frac{n)}{(n-2)!2!}
		$$
		if we craft a graph that require removing k edges to creat a min-cut we get the combination
		$$
		n\choose{k} = \frac{n)}{(n-k)!k!}
		from this it is simple to see that where k > 2
		$$
		\frac{n)}{(n-k)!k!} \leq \frac{n)}{(n-2)!2!}
		$$
		and it follow that therefor the graph that you only have to remove two edges has the most min-cuts
		So therefor there exsists no graph that has more then 
		$$
		frac{n(n-1)}{2}
		$$
		min-cut sets
	\end{proof}
	
	\bigskip
	\noindent{\bf Solution for Problem 4}
	The number of possible premutations of five letters that the monkey can type are 7,893,600. Out of those only 1 is proof
	so the probabilty of proof is type after only 5 letters is
	$$
	1/7893600
	$$
	\bigskip
	
	\noindent{\bf Solution for Problem 5}\\
	
	\bigskip
	\noindent{\bf Solution for Problem 6}\\
	
	\bigskip
	\noindent{\bf Solution for Problem 7}\\
	
	\bigskip
	\noindent{\bf Solution for Problem 8}\\
\end{document}




